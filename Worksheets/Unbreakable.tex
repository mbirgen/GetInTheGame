\worksheet{Unbreakable Tennis}
The first player to win four or more points and be ahead by two points wins the \textbf{game}. In tennis, the score begins at ``love,'' or zero points. Wining a point takes a player to 15, then 30, then 40, and then game point, which wins the game unless the players are tied 40-40, which is called a ``deuce,'' From deuce, the game proceeds until a player pulls ahead by two points.
\begin{enumerate}
	\item Serve like Agassi
	\begin{itemize}
		\item First serve: flip a coin three times.
		\begin{itemize}
			\item If exactly two tails, serve is out. Go to second serve.
			\item Else, serve is in. Play proceeds: flip a coin two times.
			\begin{itemize}
				\item If exactly two tails, Agassi loses the point.
				\item Else, Agassi winds the point.
			\end{itemize}
			\item Second serve: flip a coin one time.
			\begin{itemize}
				\item If tails, Agassi loses the point.
				\item If heads, Agassi wins the point.
			\end{itemize}
		\end{itemize}
	\end{itemize}
	Practice playing as Agassi.
	
	\begin{tabular}{*{13}{c}}
	Game  & Initials&\\
	\hline
	1 & & S: & 15 & 30 & 40 & S &S &S &S &S &S &S\\
	&&    R: & 15 & 30 & 40 & R &R &R &R &R &R &R\\ \hline
	%2 & & S: & 15 & 30 & 40 & S &S &S &S &S &S &S\\
	%&&    R: & 15 & 30 & 40 & R &R &R &R &R &R &R\\ \hline
	%3 & & S: & 15 & 30 & 40 & S &S &S &S &S &S &S\\
	%&&    R: & 15 & 30 & 40 & R &R &R &R &R &R &R\\ \hline
	%4 & & S: & 15 & 30 & 40 & S &S &S &S &S &S &S\\
	%&&    R: & 15 & 30 & 40 & R &R &R &R &R &R &R\\ \hline
	%5 & & S: & 15 & 30 & 40 & S &S &S &S &S &S &S\\
	%&&    R: & 15 & 30 & 40 & R &R &R &R &R &R &R\\ \hline
	%6 & & S: & 15 & 30 & 40 & S &S &S &S &S &S &S\\
	%&&    R: & 15 & 30 & 40 & R &R &R &R &R &R &R\\ \hline
	\end{tabular}
	
	\item Play like Sampras
	\begin{itemize}
		\item You can ask the computer to calculate a random number between 0 and 1 by typing \texttt{= RAND()} into either Excel or Google Sheets.
		\item Pick a random number between 0 and 1. If the number is less than 0.5947, Samparas's first serve goes in.
		\item If the first serve goes in, pick another uniform random number between 0 and 1. It it is less than 0.8092, then Sampras wins the point.
		\item If the first serve didn't go in, pick a uniform random number between 0 and 1 for the second serve, and if it is less than 0.5261, then Sampras wins the point.
	\end{itemize}
	Practice playing as Sampras for one game:
	
	\begin{tabular}{*{13}{c}}
	Game  & Initials&\\
	\hline
	1 & & S: & 15 & 30 & 40 & S &S &S &S &S &S &S\\
	&&    R: & 15 & 30 & 40 & R &R &R &R &R &R &R\\ \hline\end{tabular}
	
	\clearpage
	\item Now have them play each other:
	
	\begin{tabular}{*{13}{c}}
	Game  & Initials&\\
	\hline
	1 & & S: & 15 & 30 & 40 & S &S &S &S &S &S &S\\
	&&    R: & 15 & 30 & 40 & R &R &R &R &R &R &R\\ \hline
	2 & & S: & 15 & 30 & 40 & S &S &S &S &S &S &S\\
	&&    R: & 15 & 30 & 40 & R &R &R &R &R &R &R\\ \hline
	3 & & S: & 15 & 30 & 40 & S &S &S &S &S &S &S\\
	&&    R: & 15 & 30 & 40 & R &R &R &R &R &R &R\\ \hline
	4 & & S: & 15 & 30 & 40 & S &S &S &S &S &S &S\\
	&&    R: & 15 & 30 & 40 & R &R &R &R &R &R &R\\ \hline
	5 & & S: & 15 & 30 & 40 & S &S &S &S &S &S &S\\
	&&    R: & 15 & 30 & 40 & R &R &R &R &R &R &R\\ \hline
	6 & & S: & 15 & 30 & 40 & S &S &S &S &S &S &S\\
	&&    R: & 15 & 30 & 40 & R &R &R &R &R &R &R\\ \hline
	\end{tabular}
	
	\item Calculate the probability of a set reaching a 6-6 tie with neither player losing serve:
	\begin{itemize}
		\item The probability that Sampras wins one straight service game is 0.89. What is the probability that he wins 6 straight service games? \underline{\vspace{1in}}
		\item The probability that Agnassi wins one straight service game is 0.84. What is the probability that he wins 6 straight service games? \underline{\vspace{1in}}
		\item The probability that they reach 6-6 is the product of your two answers:  \underline{\vspace{1in}} What is the probability that they play four sets with neither player losing serve?  \underline{\vspace{1in}}
	\end{itemize}
\end{enumerate}